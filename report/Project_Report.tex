\PassOptionsToPackage{unicode=true}{hyperref} % options for packages loaded elsewhere
\PassOptionsToPackage{hyphens}{url}
%
\documentclass[12pt,english,]{article}
\usepackage{lmodern}
\usepackage{amssymb,amsmath}
\usepackage{ifxetex,ifluatex}
\usepackage{fixltx2e} % provides \textsubscript
\ifnum 0\ifxetex 1\fi\ifluatex 1\fi=0 % if pdftex
  \usepackage[T1]{fontenc}
  \usepackage[utf8]{inputenc}
  \usepackage{textcomp} % provides euro and other symbols
\else % if luatex or xelatex
  \usepackage{unicode-math}
  \defaultfontfeatures{Ligatures=TeX,Scale=MatchLowercase}
\fi
% use upquote if available, for straight quotes in verbatim environments
\IfFileExists{upquote.sty}{\usepackage{upquote}}{}
% use microtype if available
\IfFileExists{microtype.sty}{%
\usepackage[]{microtype}
\UseMicrotypeSet[protrusion]{basicmath} % disable protrusion for tt fonts
}{}
\usepackage{hyperref}
\hypersetup{
            pdfborder={0 0 0},
            breaklinks=true}
\urlstyle{same}  % don't use monospace font for urls
\usepackage[margin=1in]{geometry}
\usepackage{graphicx,grffile}
\makeatletter
\def\maxwidth{\ifdim\Gin@nat@width>\linewidth\linewidth\else\Gin@nat@width\fi}
\def\maxheight{\ifdim\Gin@nat@height>\textheight\textheight\else\Gin@nat@height\fi}
\makeatother
% Scale images if necessary, so that they will not overflow the page
% margins by default, and it is still possible to overwrite the defaults
% using explicit options in \includegraphics[width, height, ...]{}
\setkeys{Gin}{width=\maxwidth,height=\maxheight,keepaspectratio}
\setlength{\emergencystretch}{3em}  % prevent overfull lines
\providecommand{\tightlist}{%
  \setlength{\itemsep}{0pt}\setlength{\parskip}{0pt}}
\setcounter{secnumdepth}{0}
% Redefines (sub)paragraphs to behave more like sections
\ifx\paragraph\undefined\else
\let\oldparagraph\paragraph
\renewcommand{\paragraph}[1]{\oldparagraph{#1}\mbox{}}
\fi
\ifx\subparagraph\undefined\else
\let\oldsubparagraph\subparagraph
\renewcommand{\subparagraph}[1]{\oldsubparagraph{#1}\mbox{}}
\fi

% set default figure placement to htbp
\makeatletter
\def\fps@figure{htbp}
\makeatother

\usepackage{float}
\usepackage{indentfirst}
\usepackage[T1]{fontenc}
\usepackage{amsmath}
\usepackage{url}
\usepackage[ruled,vlined,linesnumbered]{algorithm2e}
\newcommand{\pnt}[1]{{\scriptstyle#1}}
\let\origfigure\figure
\let\endorigfigure\endfigure
\renewenvironment{figure}[1][2] {
    \expandafter\origfigure\expandafter[H]
} {
    \endorigfigure
}
\usepackage{etoolbox}
\makeatletter
\providecommand{\subtitle}[1]{% add subtitle to \maketitle
  \apptocmd{\@title}{\par {\large #1 \par}}{}{}
}
\makeatother
\ifnum 0\ifxetex 1\fi\ifluatex 1\fi=0 % if pdftex
  \usepackage[shorthands=off,main=english]{babel}
\else
  % load polyglossia as late as possible as it *could* call bidi if RTL lang (e.g. Hebrew or Arabic)
  \usepackage{polyglossia}
  \setmainlanguage[]{english}
\fi

\title{\Huge\textbf{Project Report}}
\providecommand{\subtitle}[1]{}
\subtitle{Name: Minh Thang Cao}
\author{}
\date{\vspace{-2.5em}22 June 2020}

\begin{document}
\maketitle

\newgeometry{top=1in,bottom=1in,right=1in,left=1in}

\hypertarget{introduction}{%
\section{\texorpdfstring{1
\enspace Introduction}{1 Introduction}}\label{introduction}}

Implementation is an important step of every algorithm which helps us
observe the algorithm's efficiency and behavior in practice. This report
will briefly explain each part of the algorithm, show the program's
implementation along with practical running time analysis and some
implementation techniques used. The theoretical information in this
report fully refers to the work of A. Maheshwari, W. Mulzer and M. Smid,
see {[}1{]}.

The whole closest pair algorithm consists of three important smaller
parts:

\vspace{-3.5truemm}

\begin{quote}
\begin{enumerate}
\item Computing a separating annulus, denoted \textsc{SepAnn}$(S,n,d,\mu,c)$
\item The refinement of \textsc{SepAnn}$(S,n,d,\mu,c)$, denoted \textsc{SparseSepAnn}$(S,n,d,t)$
\item The main recursive closest pair algorithm, denoted \textsc{ClosestPair}$(S,n,d)$
\end{enumerate}
\end{quote}

\vspace{-3.5truemm}

Throughout the paper, let:

\vspace{-2truemm}

\begin{quote}
\begin{itemize}
\tightlist
\item
  \((P, dist)\) be a finite metric space in which \(P\) is the set of
  all points, and \(dist\) is the function that calculate the distance
  between any two points
\item
  \(d\) be the space's doubling dimension
\item
  \(S\) be a non-empty subset of \(P\)
\end{itemize}
\end{quote}

\hypertarget{the-first-algorithm-computing-a-separating-annulus}{%
\subsection{\texorpdfstring{2 \enspace The First Algorithm: Computing a
separating
annulus}{2 The First Algorithm: Computing a separating annulus}}\label{the-first-algorithm-computing-a-separating-annulus}}

An important part of the main closest pair algorithm is finding a
separating annulus in the subset \(S\). I will briefly describe this
algorithm in the next subsection, due to A. Maheshwari, W. Mulzer and M.
Smid {[}1, Section 3.1{]}.

\hypertarget{the-mathrmspntepapntnnsndmuc-algorithm}{%
\subsection{\texorpdfstring{2.1 The
\(\mathrm{S\pnt{EP}A\pnt{NN}}(S,n,d,\mu,c)\)
algorithm}{2.1 The \textbackslash{}mathrm\{S\textbackslash{}pnt\{EP\}A\textbackslash{}pnt\{NN\}\}(S,n,d,\textbackslash{}mu,c) algorithm}}\label{the-mathrmspntepapntnnsndmuc-algorithm}}

In this section, \(\mu \ge1\) is a real constant number, \(c\) is
calculated based on \(\mu\) (I would say that \(c = 2(4\mu)^d\) {[}1,
Remark 1{]} since \(\mu\) is not an integer in this case {[}1, Section
3.2{]}).

This algorithm picks a uniformly random point \(p\) from the subset
\(S\) then finds the smallest ball centered at \(p\), denoted
\(ball_S(p, R_p)\), that contains at least \(n/c\) point. If the outer
ball \(ball_S(p,\mu R_p)\) contains at most \(n/2\) points, it returns
\(p\) and \(R_p\). If not, this procedure is repeated until the
condition is satisfied. I will rewrite this algorithm's pseudocode
below, from {[}1, Section 3.1{]}:

\medskip

\begin{figure}[ht]
  \centering
  \begin{minipage}{.8\linewidth}
    {\LinesNotNumbered
    \begin{algorithm}[H]
    \DontPrintSemicolon
    \SetAlgoLined
    \BlankLine
    \Repeat{$|ball_S(p, \mu R_p)| \geq n/c$}
      {p = a uniformly random point in $S$

      $R_p$ = min\{$r > 0: |ball_S(p, r)| \geq n/c$\}}
      return $p$ and $R_p$
    \caption{\textsc{SepAnn}$(S,n,d,\mu,c)$}
    \end{algorithm}}
  \end{minipage}
\end{figure}

\hypertarget{finding-the-kth-smallest-element}{%
\subsection{\texorpdfstring{2.2 \enspace Finding the \(K^{th}\) smallest
element}{2.2 Finding the K\^{}\{th\} smallest element}}\label{finding-the-kth-smallest-element}}

One step needed to be executed in \textsc{SepAnn($S,n,d,\mu,c$)} is to
find the smallest ball which contains at least \(n/c\) points. This ball
is easy to find using the \(k^{th}\) smallest element algorithm.
Particularly, in the list of distances between \(p\) and all other
points in \(S\), we pick the \(\lceil n/c\rceil\)-th smallest element,
and let it be the radius of the ball we need to find. Thus, all points
closer to \(p\) are inside this ball.

A very easy approach to find the \(k^{th}\) smallest element in a list
is to sort it in ascending order, and then simply return the element at
the \(k^{th}\) place. This sorting algorithm take \(O(n\,log\,n)\) time
complexity in the worst case. Fortunately, we can improve the time
complexity to \(O(n)\) using a technique which is similar to Quicksort.

\hypertarget{implementation}{%
\section{\texorpdfstring{5
\enspace Implementation}{5 Implementation}}\label{implementation}}

This implementation of the closest pair doubling algorithm of A.
Maheshwari, W. Mulzer and M. Smid {[}1{]} is written in C++ since it is
a very common and fast programming language with high level supports of
object-oriented programming that can help us organize the program
efficiently

\medskip

\begin{thebibliography}{9}
\bibitem{latexcompanion} 
A. Maheshwari, W. Mulzer and M. Smid. \emph{A Simple Randomized $O(n\,log\,n)$–Time Closest-Pair Algorithm in Doubling Metrics}, 2020. \url{https://arxiv.org/abs/2004.05883}
\end{thebibliography}

\end{document}
